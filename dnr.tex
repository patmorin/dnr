\documentclass{patmorin}
\listfiles
\usepackage{pat}
\usepackage[T1]{fontenc}
\usepackage[utf8]{inputenc}
\usepackage{mathtools}
\usepackage{thmtools}
\usepackage{thm-restate}
% \usepackage{dingbat}
% \usepackage{todonotes}
% \usepackage{comment}

\usepackage[inline]{enumitem}
\newlist{tightenum}{enumerate}{2}
\setlist[tightenum,1]{noitemsep,nosep,
                        label=\rm(\alph*),
                        ref  =\alph*}
\setlist[tightenum,2]{noitemsep,nosep,
                         label=\rm(\roman*),
                         ref  =\roman*}

\newenvironment{dummy}{}
% david proposes the following additions
% \renewcommand{\ge}{\geqslant}
% \renewcommand{\le}{\leqslant}
% \renewcommand{\geq}{\geqslant}
% \renewcommand{\leq}{\leqslant}

\newcommand{\vida}[1]{{\color{DarkGreen} Vida: #1}}
\newcommand{\pat}[1]{\textcolor{Blue}{[Pat: #1]}}
\newcommand{\gwen}[1]{\textcolor{Purple}{Gwen: #1}}
\newcommand{\piotr}[1]{\textcolor{red}{Piotr: #1}}

% \numberwithin{equation}{lem}

\crefname{p}{}{}
\creflabelformat{p}{#2(#1)#3}


\newenvironment{clmproof}{\begin{proof}[Proof of Claim:]\renewcommand{\qedsymbol}{\rule{1ex}{1ex}}}{\end{proof}}

\usepackage[longnamesfirst,numbers,sort&compress]{natbib}

% \newcommand{\mathdefin}[1]{\color{brightmaroon}#1}}
\setlength{\parskip}{1ex}

% Document-specific commands and math operators
\DeclareMathOperator{\len}{len}
\DeclareMathOperator{\sep}{sn}
\DeclareMathOperator{\tw}{tw}
\DeclareMathOperator{\pw}{pw}
\DeclareMathOperator{\bw}{bw}
\DeclareMathOperator{\td}{td}
\DeclareMathOperator{\diam}{diam}
\DeclareMathOperator{\radius}{radius}
\DeclareMathOperator{\pth}{path}
\DeclareMathOperator{\mindist}{min-dist}
\DeclareMathOperator{\mindeg}{min-deg}
\DeclareMathOperator{\girth}{girth}
\DeclareMathOperator{\dist}{dist}
\DeclareMathOperator{\ld}{ld}
\DeclareMathOperator{\polylog}{polylog}
\DeclareMathOperator{\evol}{Evol}
\DeclareMathOperator{\ivol}{Ivol}
\DeclareMathOperator{\tvol}{Tvol}
\newcommand{\NN}{\mathbb{N}}
\newcommand{\GG}{\mathcal{G}}
\newcommand{\Oh}{\mathcal{O}}
\DeclareMathOperator{\thick}{th}

\DeclarePairedDelimiter\set{\{}{\}}

\title{\MakeUppercase{Dvo\v{r}ák-Norin Revisited}}

\author{Hussein Houdrouge, Bobby Miraftab, and Pat Morin}

\date{}



\begin{document}

\maketitle

\begin{abstract}
  We give a constructive proof of the fact that the treewidth of a graph $G$ is bounded by a linear function of the separation number of $G$.
\end{abstract}


\section{Introduction}

In this paper every graph $G$ is undirected simple with vertex set $V(G)$ and edge set $E(G)$.  A \defin{separation} $(A,B)$ of a graph $G$ is a pair of subsets of $V(G)$ with $A\cup B= V(G)$ and such that, for each edge $vw$ of $G$, $\{v,w\}\subseteq A$ or $\{v,w\}\subseteq B$.  The \defin{order} of a separation $(A,B)$ is $|A\cap B|$.  A separation $(A,B)$ is \defin{balanced} if $|A\setminus B|\le \tfrac{2}{3}|V(G)|$ and $|B\setminus A|\le \tfrac{2}{3}|V(G)$.  The \defin{separation number} $\mathdefin{\sep(G)}$ of a graph $G$ is the minimum integer $a$ such that every subgraph of $G$ has a balanced separation of order at most $a$.

We reprove the following result of \citet{dvorak.norin:treewidth}
\begin{thm}
  For every graph $G$, $\tw(G)\le 15\sep(G)$.
\end{thm}



\section{Preliminaries}


A \defin{flow} $(f,g)$ in a graph $G$ is a pair of functions $f:V(G)^2\to\R_{\ge 0}$ and $g:V(G)\to\R_{\ge 0}$ such that $f(v,w)=0$ for each $vw\not\in E(G)$ and
\begin{equation}
  \sum_{w\in N_G(v)} f(v,w) + g(v) \ge \sum_{w\in N_G(v)} f(w,v)
  \label{conservation_of_flow}
\end{equation}
Any vertex $v\in V(G)$ in which \cref{conservation_of_flow} does not hold with equality is called a \defin{sink}.  The \defin{congestion} of $\mathcal{F}$ at a vertex $v\in V(G)$ the quantity on the left-hand-side of \eqref{conservation_of_flow}.  The \defin{congestion} of $\mathcal{F}$ is the maximum congestion of $\mathcal{F}$ at any vertex in $G$.  Without loss of generality we consider only flows in which at least one of $f(v,w)$ or $f(w,v)$ is zero for each edge $vw\in E(G)$.  (Otherwise, we can consider the modified flow $(f',g)$ where $f'(v,w)=f(v,w)-\min\{f(v,w),f(w,v)\}$ for each $vw\in E(G)$.)  The \defin{total supply} of $\mathcal{F}$ is $\sum_{v\in V(G)} g(v)$.


For a graph $G$ and a subset $W\subseteq V(G)$, a \defin{$W$-cloud} of $G$ is a flow $\mathcal{F}:=(f,g)$ in which $g(v)\le 1$ for each $v\in V(G)$ and the only sinks of $\mathcal{F}$ are the vertices in $W$.  \citet{dvorak.norin:treewidth} make the following observation:

\begin{obs}
  For any graph $G$ and any $W\subseteq V(G)$, there exists a $W$-cloud of $G$ with total supply $s$ and congestion at most $\alpha$ if and only if there is no separation $(X,Y)$ of $G$ with $W\subseteq Y$ such that $|Y\setminus X| + \alpha|X\cap Y| < s$.
\end{obs}

\section{The Proof}

[TODO: I didn't work out the constants. Think of something like ${?}=100$ and ${??}=10$]

\begin{lem}\label{balanced_on_w}
  Let $n\ge {?}/{??}$, let $G$ be an $n$-vertex graph, let $W\subseteq V(G)$ have size ${?}a$, let $(A,B)$ be a balanced separation of $G$ of order $a$, and let $\mathcal{F}$ be a $W$-cloud of $G$ with total supply $n$ and congestion at most $\alpha:=({??}n)/({?}a)$.  Then
  \[
  |A\cap W|,|B\cap W|\le \left(1-\frac{(1-3{??}/{?})}{3{??}}\right)?a + 1
  \]
\end{lem}

\begin{proof}
  Let $(f,g):=\mathcal{F}$.  Construct a flow $\mathcal{F'}:=(f',g')$ in which,  $g'(v):=0$, $f'(v,w):=0$, and $f'(w,v):=0$ for each $v\in A\cap B$ and each $w\in N_G(v)$.  [TODO: Explain how to reduce flow on other edges and and supply on other vertices.] In words, $\mathcal{F}'$ is obtained from $\mathcal{F}$ by preventing any flow from entering or leaving vertices in $A\cap B$.  Since each vertex in $A\cap B$ has congestion at most $\alpha$, this can be done while reducing the total supply by at most $\alpha|A\cap B|$.  The resulting flow $\mathcal{F}'$ has congestion at most $\alpha$ and total supply $n-\alpha|A\cap B|=n-\alpha a\ge (1-{??/?})n$.

  Consider the graph $G_A:=G-B$.  Since $f'(v,w)=0$ for any edge $vw$ of $G$ with an endpoint in $A\cap B$, the $W$-cloud $\mathcal{F}'$ of $G$ induces a $(W-B)$ cloud $\mathcal{F}'_A$ in $G_A$.  The total supply of $\mathcal{F'}_A$ is
  \[
     \sum_{v\in V(G_A)} g'(v) \ge |A\setminus B| -\alpha a \ge \left(\frac{n}{3}-a\right) - \frac{{??}n}{?} = \frac{(1-3{??}/?)n}{3} - a
  \]
  Since the congestion of $\mathcal{F}'$ is at most $\alpha$, the congestion of $\mathcal{F}'_A$ is at most $\alpha$.  In particular, the congestion at each vertex of $V(G_A)\cap W$ is at most $\alpha$.  Therefore,
  \[
     \alpha \ge \frac{\sum_{v\in V(G_A)} g'(v)}{|V(G_A)\cap W|}
     \ge \frac{\frac{(1-3{??}/?)n}{3} - a}{|V(G_A)\cap W|}
  \]
  Rewriting this inequality yields
  \[
    |V(G_A)\cap W|\ge \frac{(1-3{??}/?)n}{3\alpha}  - \frac{a}{\alpha}
    \ge \frac{(1-3{??}/{?})n}{3{??}n/{?}a} - \frac{?}{{??}n}
    \ge \frac{(1-3{??}/{?})?a}{3{??}} - 1
  \]
  Since $B\cap W=B\setminus (V(G_A)\cap W)$, this proves the upper bound on $|B\cap W|=|W|-|V(G_A)\cap W|={?}a-|V(G_A)\cap W|$. The upper bound on $|A\cap W|$ is proven symmetrically, by considering the graph $G_B=G-A$.
\end{proof}


\begin{lem}
  Let $a\ge 1$, let $G$ be a graph with $\sep(G)\le a$, let $W\subseteq V(G)$ have size ${?}a$, and let $(X,Y)$ be a separation of $G$ such that $W\subseteq Y$ and $|X|/|X\cap Y| > ({??}n)/({?}a)$.  Then there exists a separation $(A,B)$ of $G$ of order at most $(?a/??)+a$ such that
  \[
  |A\cap W|,|B\cap W|\le \left(\frac{1}{??}+1-\frac{(1-3{??}/{?})}{3{??}}\right)?a + 1 \enspace .
  \]
\end{lem}

\begin{proof}
  Among all separations $(X,Y)$ that satisfy the conditions of the lemma, choose one that minimizes $|Y|$.  By \cref{w_cloud_menger} $G[Y]$ has a $W$-cloud with total supply $|Y|$ and congestion at most ${??}n/{?a}$.    Let $(A',B')$ be a balanced separation of $G[Y]$ of order at most $a$.  Then, by \cref{balanced_on_w}, $|A'\cap W|\le blah$ and $|B'\cap W|\le blah$.

  We take $A:=A'\cup X$ and $B:=B'$.  Since $|X|\le n$, we have
  \[
     \frac{n}{|X\cap Y|} \ge \frac{|X|}{|X\cap Y|} \ge \frac{{??}n}{{?}a}
  \]
  which implies that $|X\cap Y|\le {?}a/{??}$.  Since $W\subseteq Y$, this implies that $|X\cap W|\le {?}a/{??}$.  Therefore, $|A\cap W|\le |X\cap W|+|A'\cap W| \le {?}a/{??} + blah$, as required.
\end{proof}


\bibliographystyle{plainurlnat}
\bibliography{dnr}


\end{document}
