\documentclass{patmorin}
\listfiles
\usepackage{pat}
\usepackage[T1]{fontenc}
\usepackage[utf8]{inputenc}
\usepackage{mathtools}
\usepackage{thmtools}
\usepackage{thm-restate}
% \usepackage{dingbat}
% \usepackage{todonotes}
% \usepackage{comment}

\usepackage[inline]{enumitem}
\newlist{tightenum}{enumerate}{2}
\setlist[tightenum,1]{noitemsep,nosep,
                        label=\rm(\alph*),
                        ref  =\alph*}
\setlist[tightenum,2]{noitemsep,nosep,
                         label=\rm(\roman*),
                         ref  =\roman*}

\newenvironment{dummy}{}
% david proposes the following additions
% \renewcommand{\ge}{\geqslant}
% \renewcommand{\le}{\leqslant}
% \renewcommand{\geq}{\geqslant}
% \renewcommand{\leq}{\leqslant}

\newcommand{\vida}[1]{{\color{DarkGreen} Vida: #1}}
\newcommand{\pat}[1]{\textcolor{Blue}{[Pat: #1]}}
\newcommand{\gwen}[1]{\textcolor{Purple}{Gwen: #1}}
\newcommand{\piotr}[1]{\textcolor{red}{Piotr: #1}}

% \numberwithin{equation}{lem}

\crefname{p}{}{}
\creflabelformat{p}{#2(#1)#3}


\newenvironment{clmproof}{\begin{proof}[Proof of Claim:]\renewcommand{\qedsymbol}{\rule{1ex}{1ex}}}{\end{proof}}

\usepackage[longnamesfirst,numbers,sort&compress]{natbib}

% \newcommand{\mathdefin}[1]{\color{brightmaroon}#1}}
\setlength{\parskip}{1ex}

% Document-specific commands and math operators
\DeclareMathOperator{\len}{len}
\DeclareMathOperator{\sep}{sn}
\DeclareMathOperator{\tw}{tw}
\DeclareMathOperator{\pw}{pw}
\DeclareMathOperator{\bw}{bw}
\DeclareMathOperator{\td}{td}
\DeclareMathOperator{\diam}{diam}
\DeclareMathOperator{\radius}{radius}
\DeclareMathOperator{\pth}{path}
\DeclareMathOperator{\mindist}{min-dist}
\DeclareMathOperator{\mindeg}{min-deg}
\DeclareMathOperator{\girth}{girth}
\DeclareMathOperator{\dist}{dist}
\DeclareMathOperator{\ld}{ld}
\DeclareMathOperator{\polylog}{polylog}
\DeclareMathOperator{\evol}{Evol}
\DeclareMathOperator{\ivol}{Ivol}
\DeclareMathOperator{\tvol}{Tvol}
\newcommand{\NN}{\mathbb{N}}
\newcommand{\GG}{\mathcal{G}}
\newcommand{\Oh}{\mathcal{O}}
\DeclareMathOperator{\thick}{th}

\DeclarePairedDelimiter\set{\{}{\}}

\title{\MakeUppercase{Dvo\v{r}ák-Norin Revisited}}

\author{Hussein Houdrouge, Bobby Miraftab, and Pat Morin}

\date{}



\begin{document}

\maketitle

\begin{abstract}
  We give a constructive proof of the fact that the treewidth of a graph $G$ is bounded by a linear function of the separation number of $G$.
\end{abstract}


\section{Introduction}

\section{Preliminaries}

In this paper any graph $G$ is simple and undirected with vertex set $V(G)$ and edge set $E(G)$.

A \defin{separation} $(A,B)$ of a graph $G$ is a pair of subsets of $V(G)$ with $A\cup B= V(G)$ and such that, for each edge $vw$ of $G$, $\{v,w\}\subseteq A$ or $\{v,w\}\subseteq B$.  The \defin{order} of a separation $(A,B)$ is $|A\cap B|$.  A separation $(A,B)$ is \defin{balanced} if $|A\setminus B|\le \tfrac{2}{3}|V(G)|$ and $|B\setminus A|\le \tfrac{2}{3}|V(G)$.  The \defin{separation number} $\mathdefin{\sep(G)}$ of a graph $G$ is the minimum integer $a$ such that every subgraph of $G$ has a balanced separation of order at most $a$.

We reprove the following result of \citet{dvorak.norin:treewidth}
\begin{thm}
  For every graph $G$, $\tw(G)\le 15\sep(G)$.
\end{thm}

A \defin{flow} $(f,g)$ in a graph $G$ is a pair of functions $f:V(G)^2\to\R_{\ge 0}$ and $g:V(G)\to\R_{\ge 0}$ such that $f(v,w)=0$ for each $vw\not\in E(G)$ and
\begin{equation}
  \sum_{w\in N_G(v)} f(v,w) + g(v) \ge \sum_{w\in N_G(v)} f(w,v)
  \label{conservation_of_flow}
\end{equation}
Any vertex $v\in V(G)$ in which \cref{conservation_of_flow} does not hold with equality is called a \defin{sink}.  The \defin{congestion} of $\mathcal{F}$ at a vertex $v\in V(G)$ the quantity on the left-hand-side of \cref{conservation_of_flow}.  The \defin{congestion} of $\mathcal{F}$ is the maximum congestion of $\mathcal{F}$ at any vertex in $G$.  Without loss of generality we consider only flows in which at least one of $f(v,w)$ or $f(w,v)$ is zero for each edge $vw\in E(G)$.  (Otherwise, we can consider the modified flow $(f',g)$ where $f'(v,w)=f(v,w)-\min\{f(v,w),f(w,v)\}$ for each $vw\in E(G)$.)  The \defin{total supply} of $\mathcal{F}$ is $\sum_{v\in V(G)} g(v)$.


For a graph $G$ and a subset $W\subseteq V(G)$, \defin{$W$-cloud} of $G$ is a flow $\mathcal{F}:=(f,g)$ in which $g(v)\le 1$ for each $v\in V(G)$ and the only sinks of $\mathcal{F}$ are the vertices in $W$.  \citet{dvorak.norin:treewidth} make the following observation:

\begin{obs}
  For any graph $G$ and any $W\subseteq V(G)$, there exists a $W$-cloud of $G$ with total supply $s$ and congestion at most $\alpha$ if and only if there is no separation $(A,B)$ of $G$ with $W\subseteq B$ such that $|B\setminus A| + \alpha|A\cap B| < s$.
\end{obs}



\section{The Proof}



\bibliographystyle{plainurlnat}
\bibliography{dnr}


\end{document}
