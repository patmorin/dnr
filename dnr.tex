\documentclass{patmorin}
\listfiles
\usepackage{pat}
\usepackage[T1]{fontenc}
\usepackage[utf8]{inputenc}
\usepackage{mathtools}
\usepackage{thmtools}
\usepackage{thm-restate}
% \usepackage{dingbat}
% \usepackage{todonotes}
% \usepackage{comment}

\usepackage[inline]{enumitem}

\newenvironment{dummy}{}
% david proposes the following additions
% \renewcommand{\ge}{\geqslant}
% \renewcommand{\le}{\leqslant}
% \renewcommand{\geq}{\geqslant}
% \renewcommand{\leq}{\leqslant}

\newcommand{\pat}[1]{\textcolor{Blue}{[Pat: #1]}}

% \numberwithin{equation}{lem}

\crefname{p}{}{}
\creflabelformat{p}{#2(#1)#3}

\usepackage{todonotes}
\newenvironment{clmproof}{\begin{proof}[Proof of Claim:]\renewcommand{\qedsymbol}{\rule{1ex}{1ex}}}{\end{proof}}

\usepackage[longnamesfirst,numbers,sort&compress]{natbib}

% \newcommand{\mathdefin}[1]{\color{brightmaroon}#1}}
\setlength{\parskip}{1ex}

% Document-specific commands and math operators
\DeclareMathOperator{\len}{len}
\DeclareMathOperator{\sep}{sn}
\DeclareMathOperator{\tw}{tw}
\DeclareMathOperator{\pw}{pw}
\DeclareMathOperator{\bw}{bw}
\DeclareMathOperator{\td}{td}
\DeclareMathOperator{\diam}{diam}
\DeclareMathOperator{\radius}{radius}
\DeclareMathOperator{\pth}{path}
\DeclareMathOperator{\mindist}{min-dist}
\DeclareMathOperator{\mindeg}{min-deg}
\DeclareMathOperator{\girth}{girth}
\DeclareMathOperator{\dist}{dist}
\DeclareMathOperator{\ld}{ld}
\DeclareMathOperator{\polylog}{polylog}
\DeclareMathOperator{\evol}{Evol}
\DeclareMathOperator{\ivol}{Ivol}
\DeclareMathOperator{\tvol}{Tvol}
\newcommand{\NN}{\mathbb{N}}
\newcommand{\GG}{\mathcal{G}}
\newcommand{\Oh}{\mathcal{O}}
\DeclareMathOperator{\thick}{th}

\DeclarePairedDelimiter\set{\{}{\}}


\newcommand{\hussein}[1]{\textcolor{purple}{HH: #1}}

\title{\MakeUppercase{Dvo\v{r}ák-Norin Revisited}}

\author{Hussein Houdrouge, Bobby Miraftab, and Pat Morin}

\date{}



\begin{document}

\maketitle

\begin{abstract}
  We give a constructive proof of the fact that the treewidth of a graph $G$ is bounded by a linear function of the separation number of $G$.
\end{abstract}


\section{Introduction}

In this paper every graph $G$ is undirected simple with vertex set $V(G)$ and edge set $E(G)$.  A \defin{separation} $(A,B)$ of a graph $G$ is a pair of subsets of $V(G)$ with $A\cup B= V(G)$ and such that, for each edge $vw$ of $G$, $\{v,w\}\subseteq A$ or $\{v,w\}\subseteq B$.  The \defin{order} of a separation $(A,B)$ is $|A\cap B|$.  A separation $(A,B)$ is \defin{balanced} if $|A\setminus B|\le \tfrac{2}{3}|V(G)|$ and $|B\setminus A|\le \tfrac{2}{3}|V(G)|$.  The \defin{separation number} $\mathdefin{\sep(G)}$ of a graph $G$ is the minimum integer $a$ such that every subgraph of $G$ has a balanced separation of order at most $a$.

We reprove the following result of \citet{dvorak.norin:treewidth}.
\begin{thm}
  For every graph $G$, $\tw(G)\le 15\sep(G)$.
\end{thm}



\section{Preliminaries}


A \defin{flow} $(f,g)$ in a graph $G$ is a pair of functions $f:V(G)^2\to\R_{\ge 0}$ and $g:V(G)\to\R_{\ge 0}$ such that $f(v,w)=0$ for each $vw\not\in E(G)$ and
\begin{equation}
  \sum_{w\in N_G(v)} f(v,w) + g(v) \ge \sum_{w\in N_G(v)} f(w,v)
  \label{conservation_of_flow}
\end{equation}
Any vertex $v\in V(G)$ in which \eqref{conservation_of_flow} does not hold with equality is called an \defin{$\mathcal{F}$-sink}.  The \defin{congestion} of $\mathcal{F}$ at a vertex $v\in V(G)$ is the quantity on the left-hand-side of \eqref{conservation_of_flow}.  The \defin{congestion} of $\mathcal{F}$ is the maximum congestion of $\mathcal{F}$ at any vertex in $G$.  The \defin{total supply} of $\mathcal{F}$ is $\sum_{v\in V(G)} g(v)$.

A \defin{pseudocycle} in a graph $G$ is a sequence of $\ell\ge 2$ vertices $v_0,\ldots,v_{\ell-1}$ of $G$ such that $v_iv_{(i+1)\bmod\ell}\in E(G)$ for each $i\in\{0,\ldots,\ell-1\}$.
Without loss of generality we consider only flows in which, for each pseudocycle $v_0,\ldots,v_{\ell-1}$ of $G$, there exists at least one $i\in\{0,\ldots,\ell-1\}$ such that $f(v_i,v_{(i+1)\bmod\ell})=0$.  (Otherwise, we can reduce $f(v_i,v_{(i+1)\bmod \ell})$ by $\min\{f(v_j,v_{(j+1)\bmod\ell}):j\in\{0,\ldots,\ell-1\}\}$ for each $i\in\{0,\ldots,\ell-1\}$.  Note that this implies that for any edge $vw$ of $G$, at least one of $f(v,w)$ or $f(w,v)$ is equal to $0$.

Any flow $\mathcal{F}:=(f,g)$ in $G$ defines a directed graph $\mathdefin{G_{\mathcal{F}}}$ that includes the directed edge $vw$ if and only if $f(v,w)>0$ and includes a vertex $v$ of $G$ if and only if $g(v)\neq 0$ or there exists an edge $vw$ of $G$ such that $f(v,w)\neq 0$ or $f(w,v)\neq 0$.  A vertex $v$ of $G_\mathcal{F}$ is \defin{$\mathcal{F}$-saturated} if $g(v)=1$, otherwise $v$ is \defin{$\mathcal{F}$-unsaturated}.  By the assumptions of the previous paragraph $G_{\mathcal{F}}$ is always acyclic.

Let $G$ be a graph and let $W$ be a subset of $V(G)$.  A \defin{$W$-cloud} of $G$ is a flow $\mathcal{F}:=(f,g)$ in which $g(v)\le 1$ for each $v\in V(G)$ and the only $\mathcal{F}$-sinks of $G$ are in $W$.  A vertex $v$ of $G_{\mathcal{F}}$ is \defin{$\mathcal{F}$-saturated} if $g(v)=1$. \citet{dvorak.norin:treewidth} make the following observation, which is a  consequence of Menger's Theorem:

\begin{obs}\label{flow_vs_cut}
  For any graph $G$ and any $W\subseteq V(G)$, there exists a $W$-cloud of $G$ with total supply $s$ and congestion at most $\alpha$ if and only if there is no separation $(X,Y)$ of $G$ with $W\subseteq Y$ such that $|Y\setminus X| + \alpha|X\cap Y| < s$.
\end{obs}

The arguments used in the proof of the following lemma are similar to those used by \citet{dvorak.norin:treewidth} in their discussion of strongly $\alpha$-tame $W$-clouds.

\begin{lem}\label{tame_w_cloud}
  If $G$ has a $W$-cloud with total supply $s\ge |W|$ and congestion $\alpha\ge 1$ then $G$ has a $W$-cloud $\mathcal{F}$ with total supply $s$ and congestion at most $\alpha$ in which the number of $\mathcal{F}$-saturated vertices in $G_\mathcal{F}$ is at least $\lfloor|V(G_{\mathcal{F}})|/2\rfloor$.
\end{lem}

\pat{We don't use the assumption $s\ge |W|$ in the proof, but we could use it to show that each vertex in $W$ is saturated.  Let's see if that's helpful later.}

\begin{proof}
  Let $\mathcal{F}$ be a $W$-cloud in $G$ with total supply $s$ and congestion $\alpha$.  Let $\sigma$ be an ordering of the vertices in $G_\mathcal{F}$ in which a vertex $u$ appears before a vertex $v$ if $G_{\mathcal{F}}$ contains a directed path from $u$ to $v$.

  Suppose that $G_{\mathcal{F}}$ contains a vertex that is unsaturated and has in-degree greater than $0$.  Let $w$ be the first such vertex that appears in $\sigma$.

  Since $G_{\mathcal{F}}$ is acyclic, it contains a directed path $v_0,\ldots,v_\ell$ from a vertex $v_0$ with in-degree $0$ to $v_\ell:=w$. Let $\Delta:=\min(\{f(v_{i-1},v_i):i\in\{1,\ldots,\ell\}\}\cup\{(1-g(w))\})$.
  For each $i\in\{1,\ldots,\ell\}$, reduce $f(v_{i-1},v_i)$ by $\Delta$, reduce $g(v_0)$ by $\Delta$, and increase $g(w)$ by $\Delta$.  This does not change the total supply, does not increase the congestion, but
  \begin{enumerate}[nosep,nolistsep]
     \item results in the removal of an edge from $G_\mathcal{F}$ (if $\Delta=f(v_{i-1},v_i)$ for some $i\in\{1,\ldots,\ell\}$); or
     \item makes $w$ a $\mathcal{F}$-saturated vertex and does not increase the number of edges in $G_\mathcal{F}$.
  \end{enumerate}
   In the second case, this operation decreases the number of vertices of $G_{\mathcal{F}}$ that have in-degree $0$ and are not $\mathcal{F}$-saturated.
   By repeating this operation as long as $G_{\mathcal{F}}$ has an vertex with in-degree $0$ that is not $\mathcal{F}$-saturated we eventually obtain a $W$-cloud with total supply $s$ and congestion at most $\alpha$ in which any vertex of $G_{\mathcal{F}}$ that is not $\mathcal{F}$-saturated has in-degree $0$.

   Now suppose that $W$ contains two unsaturated vertices $w_1$ and $w_2$. Without loss of generality, assume that $g(u_1)\ge g(u_2)$.  Let $\Delta:=\min\{g(w_2),1-g(w_1)\}$.  Decrease $w_2$ by $\Delta$ and increase $w_1$ by $\Delta$.  This either increases the number of $\mathcal{F}$-saturated vertices in $W$ (if $\Delta=1-g(w_1)$) or decreases the number of vertices in $G_\mathcal{F}$ (if $\Delta=g(w_2)$).  By repeating this operation exhaustively, we obtain a $W$-cloud in which $G_{\mathcal{F}}$ contains at most one vertex of $W$ that is not $\mathcal{F}$-saturated.

   Next, suppose that some $\mathcal{F}$-saturated vertex $v$ of $G_\mathcal{F}$ has two incoming edges $u_1v$ and $u_2v$ and neither $u_1$ nor $u_2$ is $\mathcal{F}$-saturated.  Without loss of generality, suppose that $g(u_1)\le g(u_2)$, and let $\Delta=\min\{1-g(u_1),g(u_2)$.  Decrease $g(u_2)$ and $f(u_2,v)$ by $\Delta$ and increase $g(u_1)$ and $f(u1,v)$ by $\Delta$.  This does not change the total supply of $\mathcal{F}$.  It may increase the congestion of $u_1$ but, since $u_1$ has in-degree $0$ in $G_\mathcal{F}$, the congestion at $u_1$ is at most $g(u_1)\le 1\le \alpha$.  Thus, this does not increase the congestion of $\mathcal{F}$.  On the other hand, this either
   \begin{enumerate}[nosep,nolistsep]
     \item results in the removal of a vertex and edge from $G_\mathcal{F}$ (if $\Delta=f(v_{u_2},w)$); or
     \item makes $u_1$ an $\mathcal{F}$-saturated vertex (if $\Delta=1-g(u_1)$) and does not increase the number of vertices in $G_\mathcal{F}$.
  \end{enumerate}
  In the second case, this operation increases the number of vertices of $G_{\mathcal{F}}$ that are $\mathcal{F}$-saturated.  By repeating this operation exhaustively, we obtain a $W$-cloud in which each saturated vertex $v$ in $G_{\mathcal{F}}$ has at most one incoming edge $uv$ in which $u$ is not $\mathcal{F}$-saturated.

   Now, any vertex $u$ of $G_{\mathcal{F}}$ that is not $\mathcal{F}$-saturated is either the unique such vertex in $W$ or has in-degree $0$ and is incident to an edge $uv$ where $v$ is $\mathcal{F}$-saturated. Each $\mathcal{F}$-saturated vertex $v$ is adjacent to at most one such edger $uw$. Therefore, the number of $\mathcal{F}$-saturated vertices in $G_{\mathcal{F}}$ is $k$, then $|V(G)|\le 2k+1$, and $k\ge\lfloor |V(G)|/2\rfloor$.
\end{proof}





% We will use \cref{flow_vs_cut} with $s:=n:=|V(G)|$.  In this setting,
% $|Y\setminus X|=n-|X|$. Thus, the condition $|Y\setminus X|+\alpha|X\cap Y|<s$ becomes $n-|X|+\alpha|X\cap Y|< n$.  Rewriting this inequality yields the following special case of \cref{flow_vs_cut}:

% \begin{obs}\label{flow_vs_cut_n}
%   For any $n$-vertex graph $G$ and any $W\subseteq V(G)$, there exists a $W$-cloud of $G$ with total supply $n$ and congestion at most $\alpha$ if and only if there is no separation $(X,Y)$ of $G$ with $W\subseteq Y$ such that $|X|/|Y\setminus X| > \alpha$.
% \end{obs}

\section{The Proof}

% [TODO: I didn't work out the constants. Think of something like ${C}=100$ and ${c}=10$]

% \hussein{Here, $(A, B)$ intersects $W$, because if it is not, then it contradicts $\alpha$?}
% \pat{$A\cap B$ may or may not intersect $W$.  For example, $W$ might be an independent set, in which case there is no reason $A\cap B$ needs to intersect $W$.}

% \begin{lem}\label{balanced_on_w}
%   Let $\beta >0$,
%   let $n\ge 1/\beta$, let $G$ be an $n$-vertex graph, let $W\subseteq V(G)$, let $(A,B)$ be a balanced separation of $G$ of order $a$, and let $\mathcal{F}$ be a $W$-cloud of $G$ with total supply $n$ and congestion at most $\alpha:=n/(\beta|W|)$.  Then
%   \[
%   \max\{|A\cap W|,|B\cap W|\}\le \left(1-\tfrac{\beta}{3}\right)|W|+2a \enspace .
%   \]
% \end{lem}

% \begin{proof}
%   Let $(f,g):=\mathcal{F}$.  Since the total supply of $\mathcal{F}$ is $n$, $g(v)=1$ for each $v\in V(G)$.  We construct a flow $\mathcal{F}':=(f',g')$ from $\mathcal{F}$ by preventing any flow from entering or leaving vertices in $A\cap B$.  More precisely, $g'(v):=0$, $f'(v,w):=0$, and $f'(w,v):=0$ for each $v\in A\cap B$ and each $w\in N_G(v)$.  \pat{TODO: Explain how to reduce flow on other edges and supply on other vertices to reestablish \eqref{conservation_of_flow}}.

%   Observe that the total supply in $\mathcal{F}'$ is equal to the total supply in $\mathcal{F}$' minus the total congestion at the vertices in $A\cap B$.   Since each vertex in $A\cap B$ has congestion at most $\alpha$, the total supply in $\mathcal{F}'$ is at least $n-\alpha|A\cap B|$.  Since $g'(v)\le g(v)$ for all for $v\in V(G)$, this implies that, for any $S\subseteq V(G)$,
%   \[
%     \sum_{v\in S} g'(v)\ge \sum_{v\in S}g(v) - \alpha|A\cap B| = |S| - \alpha a \enspace .
%   \]

%   Now consider the graph $G_A:=G-B$.  Since $f'(v,w)=0$ for any edge $vw$ of $G$ with an endpoint in $A\cap B$, the $W$-cloud $\mathcal{F}'$ of $G$ induces a $(W-B)$ cloud $\mathcal{F}'_A$ in $G_A$.  The total supply of $\mathcal{F}'_A$ is
%   \begin{equation}
%      \sum_{v\in V(G_A)} g'(v) \ge |A\setminus B| -\alpha a \ge \left(\frac{n}{3}-a\right) - \frac{an}{\beta|W|} \label{total_supply_a}
%      % = \left(\frac{C-c}{3C}\right)n - a = \left(\frac{1}{3}-\frac{c}{C}-\frac{a}{n}\right)n
%      % = \left(\frac{1-\epsilon}{3}\right)n
%   \end{equation}

%   Since the congestion of $\mathcal{F}'$ is at most $\alpha$, the congestion of $\mathcal{F}'_A$ is at most $\alpha$.  In particular, the congestion at each vertex of $V(G_A)\cap W$ is at most $\alpha$.  Therefore,
%   \[
%      \frac{n}{\beta|W|} = \alpha =  \ge \frac{\sum_{v\in V(G_A)} g'(v)}{|V(G_A)\cap W|}
%   \]
%   Substituting \cref{total_supply_a} and rewriting this inequality yields
%   \begin{align*}
%     |V(G_A)\cap W|
%     & \ge \left(\left(\frac{n}{3}-a\right)-\frac{an}{\beta|W|}\right)\cdot\frac{\beta|W|}{n} \\
%     & = \frac{\beta|W|}{3} - a\left(1+\frac{\beta|W|}{n}\right) \\
%     & \ge \frac{\beta|W|}{3} - 2a \enspace ,
%   \end{align*}
%   since $|W|\le n$ and $\beta\le 1$.
%   Since $B=V(G)\setminus V(G_A)$,
%   \begin{align*}
%     |B\cap W|
%       & = |(V(G)\cap W)\setminus(V(G_A)\cap W)| \\
%       & = |W|-(V(G_A)\cap W)| \\
%       & \le |W|-\frac{\beta|W|}{3} + 2a \\
%       & = \left(1-\tfrac{\beta}{3}\right)|W| + 2a \enspace . \qedhere
%   \end{align*}
% \end{proof}


% \pat{Second Try}

\begin{lem}\label{amplifier}
  Let $H$ be a graph with $\sep(H)\le a$ and let $W\subseteq V(H)$. For any integers $r\ge 0$ and $\Delta\ge 1$, there exists a separation $(P,Q)$ of $H$ of order at most $ar$ such that
  \begin{enumerate}[nosep,nolistsep,label=\rm(A\arabic*),ref=\rm A\arabic*]
    \item\label[p]{splits_w} $|P\cap W| \le \tfrac{1}{2}|W|+(a+\Delta)r$ and $|Q\cap W|\le |W|-\Delta$; or
    \item\label[p]{smaller_y} $W\subseteq Q$ and $|Q|\le (\tfrac{2}{3})^r\cdot |V(H)|+3(1-(\tfrac{2}{3})^{r})(a+\Delta) < (\tfrac{2}{3})^r\cdot |V(H)|+3(a+\Delta)$.
  \end{enumerate}
\end{lem}

\begin{proof}
  The proof is by induction on $r$. When $r=0$, taking $(P,Q)=(\emptyset,V(H))$ satisfies condition \cref{smaller_y}.  Now assume $r\ge 1$.  Let $(A,B)$ be a balanced separation of $H$ of order at most $a$.  Without loss of generality, suppose $|A\cap W|\le |B\cap W|$.  Since $|A\cap W|+|B\cap W|= |W|+|A\cap B\cap W|\le |W|+a$, we have $|A\cap W|\le \tfrac{1}{2}(|W|-a)<\tfrac{1}{2}|W|+(a+\Delta)r$.  If $|B\cap W|\le |W|-\Delta$, then the separation $(P,Q):=(A,B)$ satisfies \cref{splits_w} and we are done.

  If $|B\cap W|>|W|-\Delta$, then consider consider the graph $H':=H[B\cup W]$, which has
  \begin{align*}
    |V(H')| & = |B|+|W\setminus B| \\
      & =|B\setminus A|+|B\cap A|+|W\setminus B| \\
      & < \tfrac{2}{3}|V(H)|+a+\Delta
  \end{align*}
  vertices.  Apply the inductive hypothesis to $H'$ to obtain a separation $(P',Q')$ of $H'$ of order at most $(a+\Delta)(r-1)$, which satisifes \cref{splits_w} or \cref{smaller_y} with $r$ replaced by $r-1$.  Consider the separation $(P,Q):=(A\cup P',Q')$. The order of $(P,Q)$ is
  \begin{align*}
    |P\cap Q| & \le |A\cap Q'|+|P'\cap Q'| \\
    & \le |A\cap Q'|+ (a+\Delta)(r-1) \\
    & \le |A\cap (B\cup W)|+(a+\Delta)(r-1)\\
    & = |A\cap B| + |A\cap (W\setminus B)|+(a+\Delta)(r-1)\\
    & = |A\cap B| + |W\setminus B|+(a+\Delta)(r-1)\\
    & \le (a+\Delta)r \enspace .
  \end{align*}

  First suppose that $(X',Y')$ satisfies \cref{splits_w}.  Then
  \[
     |A\cap W| = |W|-|B\cap W|+|A\cap B\cap W|
      \le a+\Delta \enspace ,
  \]
  and
  \begin{align*}
    |P\cap W| & =|(A\cup P')\cap W| \\
     & \le |A\cap W|+|P'\cap W|  \\
     & \le (a+\Delta)+\tfrac{1}{2}|W|+(a+\Delta)(r-1)  \\
    & = \tfrac{1}{2}|W|+(a+\Delta)r \enspace .
  \end{align*}
  Since $|Y\cap W|=|Y'\cap W|\le |W|-\Delta$, the separation $(P,Q)$ satisfies \cref{splits_w}.

  If $(P',Q')$ satisfies \cref{smaller_y} then $W\subseteq Q=Q'$ and
  \begin{align*}
     |Q|=|Q'|
      & \le (\tfrac{2}{3})^{r-1}|V(H')|+3(1-(\tfrac{2}{3})^{r-1})(a+\Delta) \\
      & \le (\tfrac{2}{3})^{r-1}(\tfrac{2}{3}|V(H)|+a+\Delta)+3(1-(\tfrac{2}{3})^{r-1})(a+\Delta) \\
      & = (\tfrac{2}{3})^r\cdot|V(H)|+ (\tfrac{2}{3})^{r-1}(a+\Delta)+3(1-(\tfrac{2}{3})^{r-1})(a+\Delta) \\
      & = (\tfrac{2}{3})^r\cdot|V(H)|+ 3(1-(\tfrac{2}{3})^{r})(a+\Delta) \enspace .
  \end{align*}
  Therefore, $(P,Q)$ satisfies \cref{smaller_y}.
\end{proof}


\begin{lem}
  Let $a\ge 1$ be an integer and let $G$ be a graph with $\sep(G)\le a$.  For any $W\subseteq V(G)$, at least one of the following is true:
  \begin{enumerate}[nosep,nolistsep,label=(\alph*),ref=\alph*]
    \item\label[p]{small_w} $|W|\le c_1a$\todo{$c_1\gg 1$};
    \item\label[p]{small_y} $G$ has a separation $(X,Y)$ with $|X\cap Y|\le c_1a$, $W\subseteq Y$, and $|Y|\le c_3 a$\todo{$c_3\ge c_1$}; or
    \item\label[p]{balances_w} $G$ has a separation $(A,B)$ such that $W\neq A$, $W\neq B$, and $\max\{|A\cap (B\cup W)|,|B\cap (A\cup W)|\} \le |W|$.
  \end{enumerate}
\end{lem}

\begin{proof}
  We may assume that $|W|>c_1a$, otherwise \cref{small_w} holds and there is nothing to prove. Let $(X,Y)$ be a separation of $G$ such that
  \begin{enumerate}[nosep,nolistsep,label=(\roman*),ref=\roman*]
      \item\label[p]{has_w} $W\subseteq Y$;
      \item\label[p]{small_order} $|X\cap Y|\le c_6a$;\todo{$0<c_6<c_1$} and
      \item\label[p]{minimal_size} $|Y|$ is minimum (among all separations that satisfy \cref{has_w} and \cref{small_order}).
  \end{enumerate}
  If $|Y|\le c_3a$ then $(X,Y)$ satisfies \cref{small_y} and there is nothing more to prove.  Now assume that $|Y|> c_3a$. Let\todo{$c_7,c_8\gg 1$\newline $c_8\ge c_1$}
  \[
      s:=c_7\cdot |Y\setminus X|\text{ and let }\alpha:=\frac{c_8\cdot s}{c_1\cdot c_6\cdot a}\enspace.
  \]
  Note that $|Y\setminus X|= |Y|-|Y\cap X|\ge (c_3-c_6)a$, so
  \[
     \alpha \ge \frac{c_8\cdot c_7\cdot(c_3-c_6)}{c_1\cdot c_6} \mathcolor{red}{\ge 1} \enspace . \\
  \]
  Our intention is to use \cref{flow_vs_cut} with the parameters $s$ and $\alpha$. We first deal with the case in which $G$ has no $W$-cloud with total supply $s$ and congestion at most $\alpha$.

  Suppose that $G$ has a separation $(X',Y')$ with $W\subseteq Y'$ and $\alpha |X'\cap Y'|+|Y'\setminus X'|< s$.  This implies that $|Y'\setminus X'|<s$ and that  $\alpha|X'\cap Y'|\le s$.  The latter inequality can be rewritten as
  \[
      |X'\cap Y'|\le s/\alpha = \frac{c_1\cdot c_6\cdot a}{c_8} \enspace .
  \]
  Let $H:=G[Y']$ and observe that $W\subseteq V(H)$.
  Apply \cref{amplifier} to $H$ to obtain a separation $(P,Q)$ of order $|P\cap Q|\le ar$\todo{$1\le r\le 10$} that satifies condition \cref{splits_w} or condition \cref{smaller_y}.\todo{Choose $\Delta\ll c_1a$}

  First suppose that $(P,Q)$ satisfies \cref{smaller_y}.  We will show that, in this case, the separation $(X^\star,Y^\star):=(X'\cup P,Q)$ contradicts the choice of $(X,Y)$. By \cref{smaller_y} we have
  \begin{align*}
    |Q| & \le (\tfrac{2}{3})^r\cdot |V(H)| + 3(a+\Delta) \\
    & \le (\tfrac{2}{3})^r\cdot |Y'| + 3(a+\Delta) \\
    & \le (\tfrac{2}{3})^r\cdot (|Y'\setminus X'|+|X'\cap Y'| + 3(a+\Delta) \\
    & \le (\tfrac{2}{3})^r\cdot (|Y'\setminus X'|+\alpha|X'\cap Y'| + 3(a+\Delta) & \mathcolor{red}{\text{(since $\alpha \ge 1$)}} \\
    & \le (\tfrac{2}{3})^r\cdot s + 3(a+\Delta) \\
    & = (\tfrac{2}{3})^r\cdot c_7\cdot|Y\setminus X| + 3(a+\Delta) \\
    & \le (\tfrac{2}{3})^r\cdot c_7\cdot |Y| + 3(a+\Delta) \\
    & = |Y|-(1-\tfrac{2}{3})^r\cdot c_7\cdot|Y| + 3(a+\Delta) \\
    & < |Y|-(1-\tfrac{2}{3})^r\cdot c_7\cdot c_3\cdot a + 3(a+\Delta)
    \mathcolor{red}{\le |Y|} \enspace .
  \end{align*}
  The separation $(X^\star,Y^\star)$ has order
  \begin{align*}
    |X^\star\cap Y^\star|
      & = |(X'\cup P)\cap Q| \\
      & \le |X'\cap Q| + |P\cap Q| \\
      & \le |X'\cap Y'| + |P\cap Q| & \text{(since $Q\subseteq V(H)=Y'$)}\\
      & \le \frac{c_1\cdot c_6\cdot a}{c_8} + ar \le c_6\cdot a , \enspace
  \end{align*}
  since \textcolor{red}{$c_8\ge c_1$}.
  Thus $(X^\star,Y^\star)$ is a separation of $G$ with $W\subseteq Y^\star$, $|X^\star\cap Y^\star|\le c_6\cdot a$ and $|Y^\star|=|Q|<|Y|$, which contradicts the choice of $(X,Y)$.  We conclude that $(P,Q)$ must satisfy outcome \cref{splits_w} of \cref{amplifier}.

  We will now show that the separation $(A,B):=(X\cup X'\cup P,Q\cap Y)$ satisfies \cref{balances_w}.  This separation has order
  \begin{align*}
    |A\cap B|
    & = |(X\cup X'\cup P)\cap (Q\cap Y)| \\
    & \le |X\cap Y|+|X'\cap Q|+|P\cap Q| \\
    & \le |X\cap Y| + |X'\cap Y'| + |P\cap Q| \\
    & \le c_6\cdot a + \frac{c_1\cdot c_6\cdot a}{c_8} + ar \\
    & = \left(c_6 + \frac{c_1\cdot c_6}{c_8} + r\right)a \enspace .
  \end{align*}
  Therefore,
  \begin{align*}
    |A\cap (B\cup W)|
    & \le |A\cap B| + |B\cap W| \\
    & \le c_6\cdot a + \frac{c_1\cdot c_6\cdot a}{c_8} + ar + |W| - \Delta \\
    & \le \left(c_6 + \frac{c_1\cdot c_6}{c_8} + r\right)a + |W| - \Delta \\
    &\mathcolor{red}{\le |W|}
    % & = |(X\cup X'\cup P)\cap (Q\cap Y)| \\
    % & \le |X\cap Y|+|X'\cap Q|+|P\cap Q| \\
    % & \le |X\cap Y| + |X'\cap Y'| + |P\cap Q| \\
    % & \le c_6\cdot a + \frac{c_1\cdot c_6\cdot a}{c_8} + ar \enspace .
  \end{align*}
  A symmetric argument shows that $|B\cap (A\cup W)|\le |W|$.  Therefore, $(A,B)$ satisfies \cref{balances_w}.

  Finally, we are left with the case in which $G$ does not contain a separation $(X',Y')$ with $W\subseteq Y'$ and $\alpha|X'\cap Y'|+|Y'\setminus X'|<s$.  In this case, \cref{flow_vs_cut} implies that $G$ has a $W$-cloud $\mathcal{F}:=(f,g)$ with total supply $s$ and congestion at most $\alpha$.
  Let $n:=|V(G_{\mathcal{F}})|$.  By \cref{tame_w_cloud}, we may assume that at least $\lfloor n/2\rfloor$ vertices of  $G_{\mathcal{F}}$ are $\mathcal{F}$-saturated.
  % Therefore $\floor n/2\rfloor\le s$. The remaining $\lceil n/2\rceil$ vertices contribute less than $\lceil n/2\rceil$ to the total supply of $\mathcal{F}$.
  Since $s=\sum_{v\in V(G_{\mathcal{F}})} g(v)\le \sum_{v\in V(G_{\mathcal{F}})} 1=n$, we have $\lfloor n/2\rfloor \ge (n-1)/2 \ge (s-1)/2$.  Therefore, by removing unsaturated vertices from $\mathcal{F}$ we obtain a $W$-cloud $\mathcal{F}':=(f',g')$ with total supply $s'\ge(s-1)/2$ and congestion at most $\alpha$ in which all vertices are saturated.
  % Let $n'=s'$ be the number of vertices in $G_{\mathcal{F}'}$.
  Then\pat{I'm cheating here. I dropped the $-1/2$}
  \begin{align*}
     s' & \ge s/2 \\
        & = \frac{c_7\cdot|Y\setminus X|}{2}
  \end{align*}
  % Since all vertices in $G_{\mathcal{F}'}$ are $\mathcal{F}'$-saturated the total supply in $\mathcal{F}'$ is an integer, so the total supply of $.

  Consider the graph $G':=G[V(G_{\mathcal{F}'})\cup (Y\setminus X)]$.  Let $n':=V(G')$ be the number of vertices of $G'$.  Then
  \begin{align*}
     n' & = |V(G_{\mathcal{F}'})\cup (Y\setminus X)| \\
        & = |V(G_{\mathcal{F}'})| + |Y\setminus X\setminus V(G_{\mathcal{F}'})| \\
        & = s' + |Y\setminus X\setminus V(G_{\mathcal{F}'})| \\
        % & = s' + |Y\setminus X| \\
        % & = s' + s/c_7 \\
        % & \le s' + 2s'/c_7 \\
        % & = (1+2/c_7)s' \enspace .
  \end{align*}
  Let $(A',B')$ be a balanced separation of $G'$ of order at most $a$.  Then $|A'\setminus B'|\ge \tfrac{1}{3}n'-a$.
  \begin{align*}
    |(A'\setminus B')\cap V(G_{\mathcal{F}'})|
    & = |A'\setminus B'| - |(A'\setminus B')\setminus V(G_{\mathcal{F}'})| \\
    & = |A'\setminus B'| - |(A'\setminus B')\cap (Y\setminus X\setminus V(G_{\mathcal{F}'})| \\
    & \ge |A'\setminus B'| - |Y\setminus X\setminus V(G_{\mathcal{F}'})| \\
    & \ge \tfrac{1}{3}n' - a - |Y\setminus X\setminus V(G_{\mathcal{F}'})| \\
    & = \tfrac{1}{3}s' - a - \tfrac{2}{3}|Y\setminus X\setminus V(G_{\mathcal{F}'})| \\
    & \ge \tfrac{1}{3}s' - a - \tfrac{2}{3}|Y\setminus X| \\
    & \ge \frac{c_7\cdot|Y\setminus X|}{6} - a - \tfrac{2}{3}|Y\setminus X| \\
    & = \frac{(c_7-4)\cdot |Y\setminus X|}{6} - a \\
  \end{align*}
  Construct a flow $\mathcal{F}'':=(f'',g'')$ from $\mathcal{F}'$ in which each vertex $v$ in $A'\cap B'$ is a sink.  We do this by setting $f''(v,w)=0$ for each vertex $v$ in $A'\cap B'$ and each $w\in N_{G'}(w)$, and reducing flows on other edges in order to satisfy \eqref{conservation_of_flow}.  For each edge $vw$ of $G$, $f''(v,w)\le f'(v,w)\le f(v,w)$. Therefore, the congestion of $\mathcal{F}''$ is at most the congestion of $\mathcal{F}$, which is at most $\alpha$.  Therefore, the sinks of $\mathcal{F}''$ in $A'\cap B'$ consume\todo{define consume} at most $\alpha|A'\cap B'|\le \alpha a$. Therefore, in $\mathcal{F}''$, the total supply originating at vertices in $A'\setminus B'$ that is consumed by vertices in $(A'\setminus B')\cap W$ is at least
  \[
    |(A'\setminus B')\cap V(G_{\mathcal{F}'})| - \alpha a
    \ge \frac{(c_7-4)\cdot |Y\setminus X|}{6} - (\alpha+1)a
  \]
  Since the congestion of $\mathcal{F}''$ is at most $\alpha$, each vertex in $(A'\setminus B')\cap W$ consumes at most $\alpha$, so
  \[
    \frac{(c_7-4)\cdot |Y\setminus X|}{6} - (\alpha+1)a \le \alpha\cdot|(A'\setminus B')\cap W| \enspace .
  \]
  Rewriting this gives
  \begin{align*}
    |A'\cap W|\ge
   |(A'\setminus B')\cap W|
    & \ge \frac{(c_7-4)\cdot |Y\setminus X|}{6\alpha} - \frac{(\alpha+1)a}{\alpha} \\
    & = \frac{(c_7-4)\cdot c_1\cdot c_6\cdot a}{6\cdot c_8\cdot c_7} - \frac{(\alpha+1)a}{\alpha}
  \end{align*}
  Therefore,
  \begin{align*}
   |(B'\setminus A')\cap W|
    & = |W|-|A'\cap W| \\
    & \le |W|- \frac{(c_7-4)\cdot c_1\cdot c_6\cdot a}{6\cdot c_8\cdot c_7} + \frac{(\alpha+1)a}{\alpha}
  \end{align*}


\end{proof}





% \begin{lem}
%   Let $a\ge 1$, let $G$ be a graph with $\sep(G)\le a$, let $W\subseteq V(G)$, and let $(X,Y)$ be a separation of $G$ such that $W\subseteq Y$ and $|X|/|X\cap Y| > n/(\beta|W|)$.  Then there exists a separation $(A,B)$ of $G$ of order at most $\beta|W|+a$ such that
%   \[
%   \max\{|A\cap W|,|B\cap W|\}\le \left(1-\tfrac{\beta}{3}+\tfrac{\beta^2}{3}\right)|W| \enspace .
%   \]
% \end{lem}

% \begin{proof}
%   Among all separations $(X,Y)$ that satisfy the conditions of the lemma, choose one that maximizes $|X|$.    Let $W':=W\setminus X$.
%   Create a new graph $G'\supseteq G[Y\setminus X]$ as follows:  Let
%   \[
%     n':=n\left(1-\frac{|W\cap X|}{\beta|W|}\right) \enspace .
%   \]
%   Add a set $L$ of $n'-|Y\setminus X|$ degree-$1$ vertices to $G[Y\setminus X]$ adjacent to the vertices in $N_G(X)$, and distributed as equally as possible.  For each $v\in N_G(X)$, let $L_v$ be the subset of $L$ adjacent to $v$.

%   Let
%   \[
%     \beta' = \frac{\beta|W|n'}{|W'|n}
%   \]
%   We will show that $G'$ has a $W'$-cloud with total supply $n'$ and congestion at most $n'/(\beta'|W'|)=n/(\beta|W|)$.  By \cref{flow_vs_cut_n}, it suffices to show that $G'$ has no separation $(X',Y')$ with $W'\subseteq Y'$ and $|X'|/|X'\cap Y'|>n'/(\beta|W'|)$.  Assume for the sake of contradiction that such a separation exists and, among all such separations, choose one that maximizes $|X'|$ and, subject to this, minimizes $|Y'|$.

%   \begin{clm}\label{no_l_in_y}
%     $L\cap Y'=\emptyset$.
%   \end{clm}

%   \begin{clmproof}
%     Let $v$ be a vertex in $N_G(X)$. Since $L=\bigcup_{v\in N_G(X)}L_v$, we need only show that $L_v\cap Y'=\emptyset$.
%     Let $v'$ be an arbitrary vertex in $L_v$.  If $\{v,v'\}\subseteq X'$, then the separation $(X',Y'\setminus\{v'\})$ contradicts the minimality of $Y'$ in the choice of $(X',Y')$.  If $v\in X'$ and $v'\not\in X'$ then the separation $(X'\cup\{v'\},Y'\setminus\{v'\})$ contradicts the maximality of $X'$ in the choice of $(X',Y')$.

%     Thus, $v\not\in X'$.  Let $X'':=X'\setminus L_v$, consider the separation $(X'',Y')$ and observe that
%     \[
%       \frac{X''}{|X''\cap Y'|}=\frac{|X'\setminus L_v|}{|(X'\setminus L_v)\cap Y'|}
%      = \frac{|X'|-|X'\cap L_v|}{|X'\cap Y'|-|X'\cap L_v|} > \frac{n'}{\beta|W'|}
%     \]
%     since $(a-x)/(b-x) \ge a/c$ for any $a,b,x$ with $0\le x<b<a$.  Then the separation $(X''',Y'''):=(X''\cup L_v\cup \{v\},Y'\setminus L_v)$ satisfies
%     \[
%       \frac{|X'''|}{|X'''\cap Y'''|} =
%       \frac{|X''\cup L_v\cup \{v\}|}{|(X''\cup L_v\cup\{v\})\cap (Y'\setminus L_v)|} = \frac{|X''|+n'/(\beta|W'|)+1}{|X''\cap Y'|+1} > \frac{n'}{\beta|W'|}
%     \]
%     since $(a+m)/(b+1)\ge m$ for any $m>0$ and any $a,b>0$ such that $a/b\ge m$.  Therefore, the separation $(X''',Y''')$ contradicts the maximality of $|X'|$ in the choice of $(X',Y')$.
%   \end{clmproof}

%   \begin{clm}
%     $N_G(X)\subseteq X'$.
%   \end{clm}

%   \begin{clmproof}
%     It must be the case that $N_{G'}[L]\subseteq X'$ since, otherwise, there would be an edge $vv'$ of $G'$ with $v\in N_{G'}(L)$ and $v'\in L$ such that $\{v,v'\}\not\subseteq X'$ and $\{v,v'\}\not\subseteq Y'$.  This would contradict the fact that $(X',Y')$ is a separation of $G$.
%     Therefore $N_G(X)\subseteq N_{G'}(L)\subseteq X'$.
%   \end{clmproof}

%   Let
%   \[
%     \widehat{X}:=X\cup X'\setminus L \text{ and let } \widehat{Y}:=Y'\cup W \enspace .
%   \]
%   \begin{clm}
%     $(\widehat{X},\widehat{Y})$ is a separation of $G$.
%   \end{clm}

%   \begin{clmproof}
%     We first show that $\widehat{X}\cup\widehat{Y}\subseteq V(G)$.
%     The vertices in $\widehat{X}\cup\widehat{Y}$ are contained $X\cup V(G')\subseteq V(G)\cup L$.
%     By definition $L\cap \widehat{X}=\emptyset$. By \cref{no_l_in_y}, $L\cap \widehat{X}=\emptyset$. Thus $\widehat{X}\cup\widehat{Y}\subseteq V(G)$.
%     Next we show that $\widehat{X}\cup\widehat{Y}\supseteq V(G)$.  Since $X\cup Y=V(G)$, each vertx of $G$ is in $X$ or is in $Y\setminus X$.
%     For each $v\in X$, $v\in \widehat{X}$.  For each $v\in Y\setminus X$,
%     \begin{enumerate*}[label=(\alph*)]
%       \item $v\in X'$, in which case $v\in \widehat{X}$;
%       or \item $v\in Y'$ in which case $v\in\widehat{Y}$.
%     \end{enumerate*}
%     Thus $V(G)\subseteq \widehat{X}\cup\widehat{Y}\subseteq V(G)$, so $\widehat{X}\cup\widehat{Y}= V(G)$.

%     What remains is to verify that for each edge $vw$ of $G$, $\{v,w\}\subseteq \widehat{X}$ or $\{v,w\}\subseteq\widehat{Y}$.
%     Without loss of generality (at least) one of the following applies:
%     \begin{enumerate}[nosep,nolistsep,label=(\roman*)]
%       % \item $v\in X\cap W$: In this case $v\in\widehat{X}$ and $v\in\widehat{Y}$.
%       \item $v\in X$: In this case $v\in\widehat{X}$ and $w\in N_G[X]\subseteq\widehat{X}$.
%       \item $v\in N_G(X)$: In this case $w\in X$ or $w\in Y\setminus X$.  In the former case $v\in N_G(X)\subseteq X$ and $w\in X\subseteq \widehat{X}$. In the latter case $v,w\in Y\setminus X$.  Since $(X',Y')$ is a separation of $G'\supseteq G[Y\setminus X]$, either $v,w\in X'\subseteq \widehat{X}$ or $v,w\in Y'\subseteq\widehat{Y}$. \qedhere
%     \end{enumerate}
%   \end{clmproof}

%   Next we show that $(\widehat{X},\widehat{Y})$ contradicts the maximality of $X$ in the choice of $(X,Y)$.

%   \begin{clm}
%     $|\widehat{X}|/|\widehat{X}\cap\widehat{Y}|>n/(\beta|W|)$.
%   \end{clm}

%   \begin{clmproof}
%     Since $Y'\subseteq Y\setminus X$, $X\cap \hat{Y}=X\cap W$. Since $X'\subseteq Y\setminus X$,
%     \[
%       X'\cap \widehat{Y}=X'\cap (Y'\cup W)=X'\cap (Y'\cup (W\cap X))=X'\cap Y' \enspace .
%     \]
%     Therefore, $|\widehat{X}\cap\widehat{Y}|=|X'\cap Y'|+|X\cap W|$.
%     On the other hand,
%     \begin{align*}
%       |\widehat{X}|
%         & = |X| + |X'| - |L| \\
%         & = |X| + |X'| - (n'-|Y\setminus X|) \\
%         & = |X| + |X'| - (n'-(n-|X|)) \\
%         & = |X'| - (n'-n)) \\
%         & = |X'| + \frac{|W\cap X|n}{\beta|W|} \\
%     \end{align*}
%     Now recall that
%     \begin{align*}
%       \frac{|X'|}{|X'\cap Y'|}
%         & >  \frac{n'}{\beta'|W'|} = \frac{n}{\beta|W|} \enspace .
%     \end{align*}


%     Therefore,
%     \begin{align*}
%       \frac{|\widehat{X}|}{|\widehat{X}\cap\widehat{Y}|}
%         & = \frac{|X'| + \frac{|W\cap X|n}{\beta|W|}}{|X'\cap Y'|+|X\cap W|}  \\
%         & > \frac{n}{\beta|W|} \qedhere
%     \end{align*}
%   \end{clmproof}
%   Since $|\widehat{X}|>|X|$, this contradicts the maximality of $|X|$ in the choice of $(X,Y)$

%   Thus $G'$ has a $W'$-cloud with total supply $n'$ and congestion at most $n'/(\beta'|W|)=n/(\beta|W|)$.  Let $(A',B')$ be a balanced separation of $G[X\setminus Y]$ of order at most $a$.  For each vertex $v\in V(G')\setminus V(G)$, add $v$ to $A'$ if the unique neighbour of $v$ is in $A'$, otherwise add $v$ to $B'$.  This gives separation of $G'$ whose order is at most $a$.  \textcolor{red}{[There's a big problem here.  We can't be sure that the resulting separation is a balanced separation of $G'$! This means we can't apply \cref{balanced_on_w}]}  By \cref{balanced_on_w}, $|A'\cap W\setminus X|\le (1-\tfrac{\beta'}{3})|W\setminus X|+2a$ and $|B'\cap W\setminus X|\le (1-\tfrac{\beta'}{3})|W\setminus X|+2a$.

%   Finally, let $A:=X\cup A'\cap V(G)$ and let $B:=(X\cap Y)\cup B'\cap V(G)$.  Then $(A,B)$ is a separation of $G$ with
%   \[
%     |A\cap B|\le |X\cap Y|+|A'\cap B'| \le \beta|W| + a
%   \]
%   Furthermore,
%   \begin{align*}
%     |A\cap W| & \le |X\cap W|+(1-\tfrac{\beta'}{3})|W\setminus X| \\
%        & = |X\cap W|+(1-\tfrac{\beta'}{3})(|W|-|W\cap X|) \\
%        & = \tfrac{\beta'}{3}|X\cap W|+(1-\tfrac{\beta}{3})|W| \\
%        & \le \tfrac{\beta\beta'}{3}|W|+(1-\tfrac{\beta'}{3})|W| \\
%        & = (1-\tfrac{\beta}{3}+\tfrac{\beta\beta'}{3})|W|
%   \end{align*}
% \end{proof}



\bibliographystyle{plainurlnat}
\bibliography{dnr}


\end{document}
